\documentclass[9pt]{article}
\usepackage{fullpage}
\usepackage{amsmath}
\usepackage{amssymb}
\usepackage[usenames]{color}

\leftmargin=0.25in
\oddsidemargin=0.25in
\textwidth=6.0in
\topmargin=-0.25in
\textheight=9.25in

\raggedright

%\pagenumbering{arabic}
\pagenumbering{gobble}

\def\bull{\vrule height 0.8ex width .7ex depth -.1ex }
% DEFINITIONS FOR RESUME

\newenvironment{changemargin}[2]{%
  \begin{list}{}{%
    \setlength{\topsep}{0pt}%
    \setlength{\leftmargin}{#1}%
    \setlength{\rightmargin}{#2}%
    \setlength{\listparindent}{\parindent}%
    \setlength{\itemindent}{\parindent}%
    \setlength{\parsep}{\parskip}%
  }%
  \item[]}{\end{list}
}

\newcommand{\lineover}{
	\begin{changemargin}{-0.05in}{-0.05in}
		\vspace*{-8pt}
		\hrulefill \\
		\vspace*{-2pt}
	\end{changemargin}
}

\newcommand{\header}[1]{
	\begin{changemargin}{-0.5in}{-0.5in}
		\scshape{#1}\\
  	\lineover
	\end{changemargin}
}

\newcommand{\contact}[4]{
	\begin{changemargin}{-0.5in}{-0.5in}
		\begin{center}
			{\Large \scshape {#1}}\\ \smallskip
			{#2}\\ \smallskip 
			{#3}\\ \smallskip
			{#4}\smallskip
		\end{center}
	\end{changemargin}
}

\newenvironment{body} {
	\vspace*{-16pt}
	\begin{changemargin}{-0.25in}{-0.5in}
  }	
	{\end{changemargin}
}	

\newcommand{\school}[4]{
	\textbf{#1} \hfill \emph{#2\\}
	#3\\ 
	#4\\
}

% END RESUME DEFINITIONS

\begin{document}

%%%%%%%%%%%%%%%%%%%%%%%%%%%%%%%%%%%%%%%%%%%%%%%%%%%%%%%%%%%%%%%%%%%%%%%%%%%%%%%%
% Contact Info
\begin{center} 
	{\Large \scshape Vibhor Kumar} 
\end{center}
	\vspace*{-2pt}
\hspace{-12mm} Github: \hspace{3mm} \textbf{https://github.com/veezbo/}
\hspace{46mm} Email: \hspace{0.1mm} \textbf{vibhor@caltech.edu} \\
\vspace*{1pt}
\hspace{-12mm} LinkedIn: \hspace{0.25mm} \textbf{https://linkedin.com/in/kumarvibhor/} \hspace{26.75mm} Phone: \textbf{(513) 293-1138}\\
%\hspace{-6mm} Github: \textbf{github.com/veezbo} \hspace{9mm} Email: \textbf{vibhor@caltech.edu} \hspace{9mm} Phone: \textbf{(513) 293-1138}
\vspace{1pt}
\smallskip
%%%%%%%%%%%%%%%%%%%%%%%%%%%%%%%%%%%%%%%%%%%%%%%%%%%%%%%%%%%%%%%%%%%%%%%%%%%%%%%%
% Education
\header{Education}
%\smallskip
\begin{body}
	\vspace{15pt}
	\textbf{California Institute of Technology [Caltech]}{}, \emph{Computer Science, 3.8 GPA} \hfill \emph{Class of 2016}{} \\
	\begin{itemize} \itemsep -0pt
	\item CG Research for Credit (CS 81), Computer Graphics (CS 171), GPU Programming (CS 179), Computer Graphics Research Project class (CS 174), Introduction to Computer Graphics Research (CS 176), Discrete Differential Geometry (CS 177), Machine Learning (CS 156ab), Data Structures and Algorithms (CS 38), Decidability and Tractability (CS 21), Introduction to Computing Systems (CS 24)
	\end{itemize}
\end{body}

\smallskip

%%%%%%%%%%%%%%%%%%%%%%%%%%%%%%%%%%%%%%%%%%%%%%%%%%%%%%%%%%%%%%%%%%%%%%%%%%%%%%%%
% Experience
\header{Work and Experience}
%\smallskip
\begin{body}
	\vspace{15pt}
	\textbf{Apple}, \emph{Platform Architecture - Graphics Research Intern} \hfill \emph{June 2014 - September 2014}\\
	\vspace*{-3pt}
	\begin{itemize} \itemsep -0pt %reduce space between items
	\item Worked with the exploratory graphics team to provide recommendations for future graphics hardware.
	\item Developed novel graphics software algorithms for efficient and visually appealing improvements over the current pipeline, and pre-existing techniques.
	\end{itemize}
	\vspace*{-3pt}
	\textbf{Caltech}, \emph{Teaching Assistant for CS 171: Computer Graphics} \hfill \emph{September 2013 - January 2014}\\
	\vspace*{-3pt}
	\begin{itemize} \itemsep -0pt %reduce space between items
	\item Responsibilities included holding occasional recitation sessions, weekly office hours, and grading student assignments.
	\end{itemize}
	\vspace*{-3pt}
	\textbf{Samsung Mobile R\&D}, \emph{Software Engineering Intern} \hfill \emph{July 2013 - September 2013}\\
	\vspace*{-4pt}
	\begin{itemize} \itemsep -0pt %reduce space between items
	\item Developed testing suites for enterprise security applications and their derivatives by working with Android testing tools and bash/python scripting for full automation.
	\item Worked alongside my team by handling automation and development, and created test cases for various parts of the overall testing suite, including optimizing a UI interface for internal use.
	\item Fixed errors with testing tools, involving careful manipulation of the Android source code, while additionally ensuring compatibility with Samsung's current and future internal modifications.
	\end{itemize}
	\vspace*{-3pt}
	\textbf{ChoiceFork}, \emph{Researcher} \hfill \emph{April 2013 - June 2013}\\
	\vspace*{-4pt}
	\begin{itemize} \itemsep -0pt  % reduce space between items
		\item Building up from scratch the logistics of improving decision making on the internet as the sole researcher and developer. Ideas were centered around Natural Language Processing.
	\end{itemize}
\smallskip
	\vspace*{-5pt}
	\textbf{Vora Labs}, \emph{Testing and Software Development} \hfill \emph{Summer 2012}\\
	\vspace*{-5pt}
	\begin{itemize} \itemsep -0pt  % reduce space between items
		\item Created unit tests with JUnit for all parts of the development on the backend, while additionally developing various backend and frontend tasks as needed, including product search recommendations.
	\end{itemize}
	\vspace*{-3pt}
\smallskip

	
\end{body}

% \newpage{} % uncomment this line if you want to force a new page



%%%%%%%%%%%%%%%%%%%%%%%%%%%%%%%%%%%%%%%%%%%%%%%%%%%%%%%%%%%%%%%%%%%%%%%%%%%%%%%%
% Projects
\header{Selected Research Projects}
%\smallskip
\begin{body}
	\vspace{14pt}
	\textbf{Boundary Conditions of Mesh Conformal Parametrization}, \emph{CS176 Research} \hfill \emph{Dec. 2013 - Present} \\
	\begin{itemize} \itemsep -0pt
	\item Investigating boundary conditions involved in conformal parametrization of meshes, and finding alternatives to the de-facto Von Neumann boundary conditions in mesh processing.
	\end{itemize}
	
	\textbf{Edge-Preserving Linear-Time Smoothing Optimization}, \emph{CS174 Research} \hfill \emph{March 2013 - June 2013} \\
	\begin{itemize} \itemsep -0pt
		\item Optimizing an edge-preserving linear-time smoothing filter algorithm (Gastal, Oliviera; SIGGRAPH 2011), and implementing both a CPU and GPU version with novel improvements.
	\end{itemize}
	\textbf{Automated Inbetweening between Noisy Keyframes}, \emph{CS81 Research} \hfill \emph{Dec. 2012 - March 2013} \\
	\begin{itemize} \itemsep -0pt
		\item Using ideas from both Computer Graphics and Applied Math to develop a generalized method that solves the research problem and is faithful to the artist's intent.
	\end{itemize}

\end{body}

\smallskip

%%%%%%%%%%%%%%%%%%%%%%%%%%%%%%%%%%%%%%%%%%%%%%%%%%%%%%%%%%%%%%%%%%%%%%%%%%%%%%%%
% Skills
\header{Skills and Awards}
\smallskip
\begin{body}
	\vspace{14pt}
	
	\begin{itemize} \itemsep -0pt
	
	\item \textbf{Skills:} C/C++, Java, Python, OpenGL, JOGL and PyOpenGL, CUDA, GLSL, Android (Eclipse/ADT), Bash Scripting, \LaTeX, Markdown, ANTLR,  Mathematica, R, Octave, SVN, Git and Github ( \textbf{see above} )\\
	\medskip
	\item \textbf{Awards:} Stauffer Scholarship (Caltech, \emph{Winter-Spring 2014}), Marion Gene Vincenti Scholarship (Caltech, \emph{Spring 2013}),
        Frances and Howard Vesper Scholarship (Caltech, \emph{Spring 2013}),
        1st place Google Games Coding Competition (\emph{Spring 2013}),
        11th Place Regional ACM Collegiate Programming Contest (\emph{Winter 2012}),
        Association for Computing Machinery Membership, (\emph{Spring 2013}),
        Top 5 in Silver Division on USA Computing Olympiad US Open Contest (\emph{Spring 2012}),
        Honda-OSU Math Medal (\emph{Spring 2012}),
        1st Place American Computer Science League Contest (\emph{Spring 2012}).
	\end{itemize}
\end{body}

\end{document}
